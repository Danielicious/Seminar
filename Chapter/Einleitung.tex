\addcontentsline{toc}{chapter}{Einleitung}
\chapter*{Einleitung}
%In den Jahrzehnten der Historie von Softwareentwicklung gab es immer wieder neue Erkenntnisse und neue State-of-the-Art Methoden der Entwicklung von Software. So gab es lange Zeit große, monolithische Desktop Applikationen, welche nur als großes Ganzes funktionierten. Mittlerweile geht der Trend hin zu Microservices\footnote{Dabei handelt es sich um eine Zusammenstellung unabhängiger Prozesse, die durch eine sehr leichtgewichtige Kommunikationsschicht verbunden sind. Ein Beispiel zur Kommunikation ist HTTP(S). Weitergehende Informationen z.B. unter \cite{fowler-Microservice}}. Diese, bis auf die Schnittstellenbeschreibung, unabhängige Entwicklung der einzelnen Komponenten von Software, erlaubt eine wesentlich schnellere Erstellung von Software.\\
%Auch die Einstellung der Nutzer hat sich verändert. Durch das schnelle Entwickeln von Patches und Updates, ist der Anwender zum Beta Tester von Software avanciert. Er ist es nicht nur gewohnt, dass in schneller Abfolge neue Änderungen an der Software veröffentlicht werden, sondern er erwartet es regelrecht.\\
%Ein weiterer Aspekt sind die neuen Vorgehensmodelle im (Software-)Projektmanagement. In der Vergangenheit war es gang und gäbe, das Wasserfall Modell zu verwenden. Dabei wird der Test in einer späten Projektphase durchgeführt, und die Entwicklungsphase dauert sehr lange, bis das Gesamtprodukt fertig entwickelt ist. Heutzutage bedient man sich eher agiler Modelle wie z.B. Scrum. Hierbei wird in  regelmäßigen Abständen eine überschaubare Verbesserung des Produkts veröffentlicht. Dies unterstützt die oben beschriebene Veränderung in modernen Software Architekturen.\\
%In diesen Zeiten immer kürzerer Entwicklungszyklen gewinnt die Entwicklung von Konzepten zur Sicherung der Code Qualität zunehmend an Bedeutung. Eines dieser Konzepte, das ich in der hier vorliegenden Seminararbeit näher beleuchten möchte, ist \textbf{Continuous Integration}.\\
%Software soll schnell entwickelt und getestet werden. Dies ist nur durch eine weitreichende Automatisierung von Build-, Integrations- und Testschritten möglich. Mehrere kommerzielle und kostenlose Tools zur Unterstützung von Continuous Integration existieren am Markt, wobei diese Arbeit \textbf{Jenkins} genauer vorstellt.

In den Jahrzehnten der Historie von Softwareentwicklung gab es immer wieder neue Erkenntnisse zur Entwicklung von Software. So gab es lange Zeit große, monolithische Desktop Applikationen, welche nur als großes Ganzes funktionierten. Mittlerweile ist die Aufteilung der Software in einzelne Bereiche Standard, da eine Wartung sonst kaum effizient möglich ist. Im Bereich Cloud geht der Trend hin zu Microservices\footnote{Dabei handelt es sich um eine Zusammenstellung unabhängiger Prozesse, die durch eine sehr leichtgewichtige Kommunikationsschicht verbunden sind. Ein Beispiel zur Kommunikation ist HTTP(S). Weitergehende Informationen z.B. unter \cite{fowler-Microservice}}.\\
Diese, bis auf die Schnittstellenbeschreibung, unabhängige Entwicklung der einzelnen Komponenten von Software, erlaubt eine sehr entkoppelte Erstellung. Die immer schnellere Abfolge von Softwarereleases ist nur durch größere Teams zu bewerkstelligen. Die gesamte Architektur wird modularisiert und dedizierte Teams gebildet die für bestimmte Teile der Software verantwortlich sind. Dadurch gewinnt die Integration der einzelnen Teile eine sehr zentrale Bedeutung.\\
Ein weiterer Aspekt sind die neuen Vorgehensmodelle im (Software-)Projektmanagement. In der Vergangenheit war es gang und gäbe, das Wasserfall Modell zu verwenden. Dabei wird der Test in einer späten Projektphase durchgeführt, und  die Entwicklungsphase dauert sehr lange, bis das Gesamtprodukt fertig entwickelt ist. Heutzutage bedient man sich eher agiler Modelle wie z.B. Scrum. Hierbei wird in regelmäßigen Abständen eine überschaubare Verbesserung des Produkts erreicht. Dies unterstützt die oben beschriebene Veränderung in modernen Software Architekturen. \\
In diesen Zeiten immer kürzerer Entwicklungszyklen gewinnt die Entwicklung von Konzepten zur Sicherung der Code Qualität zunehmend an Stellenwert. Eines dieser Konzepte, das in der vorliegenden Seminararbeit näher beleuchtet werden soll, ist \textbf{Continuous Integration}.\\
 Software soll schnell entwickelt und getestet werden. Dies ist nur durch eine weitreichende  Automatisierung von Build-, Integrations- und Testschritten möglich. Am Markt existieren mehrere kommerzielle und kostenlose Tools zur Unterstützung von Continuous Integration, wobei diese Arbeit \textbf{Jenkins} genauer vorstellt.
