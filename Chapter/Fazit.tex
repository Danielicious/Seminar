\pagebreak

\chapter{Fazit}
Continuous Integration ist eine sinnvolle Ergänzung zu allen Softwareentwicklungs Projekten. Dem vergleichsweise geringen Aufwand für eine erste Realisierung des Konzepts im Unternehmen steht ein hoher Vorteil gegenüber. Die Software selbst wird stabiler, das Vertrauen in das eigene Produkt wächst und die Nachvollziehbarkeit der Ursache von Problemen steigt immens.
Aufgrund der sehr guten Tool Unterstützung die für nahezu jeden Anwendungsfall bereits eine passende Lösung bietet ist es selbst für Entwickler die in CI noch nicht so sehr bewandert sind möglich, das Konzept umzusetzen.
Es ist uneingeschränkt empfehlenswert CI selbst in kleinen Projekten zu implementieren, da der Nutzen nicht von der Hand zu weisen ist und ihm nur ein überschaubarer Aufwand gegenübersteht.
