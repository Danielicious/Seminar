\pagebreak

\chapter{Fazit}
Continuous Integration ist eine sinnvolle Ergänzung zu allen Softwareentwicklungs Projekten. Dem vergleichsweise geringen Aufwand für eine erste Realisierung des Konzepts im Unternehmen steht ein hoher Vorteil gegenüber. Die Software selbst wird stabiler, das Vertrauen in das eigene Produkt wächst und die Nachvollziehbarkeit der Ursache von Problemen steigt immens.
Aufgrund der sehr guten Tool Unterstützung die für nahezu jeden Anwendungsfall bereits eine passende Lösung bietet ist es selbst für Entwickler die in CI noch nicht so sehr bewandert sind möglich, das Konzept umzusetzen.
Es ist uneingeschränkt empfehlenswert CI selbst in kleinen Projekten zu implementieren, da der Nutzen nicht von der Hand zu weisen ist und ihm nur ein überschaubarer Aufwand gegenübersteht.\\
Jenkins als Vertreter dieser Tools ist ein sehr mächtiges Werkzeug das eine immense Entwicklergemeinde hinter sich weiß. Es gibt für unheimlich viele Aufgaben bereits vorhandene Plugins die das System an die konkreten Bedürfnisse anpassen. Auch aufgrund des Fakts, dass das Tool kostenlos ist, und der sehr guten Erweiterungsmöglichkeiten, hält der Autor Jenkins für uneingeschränkt empfehlenswert.\\
In seinem Alltag ist der Autor Build- und Configurationmanager bei einem großen Unternehmen. Dort ist bereits der Übergang zu CD im vollen Gange. Besonders weil auch regulatorische Aspekte dort eine Rolle spielen, hat sich die Nachvollziehbarkeit und Vermeidung von menschlichen Fehlern durch Automatisierung als sehr vorteilhaft herausgestellt. Es ist jedoch immer ein Abwägen nötig, ob Kosten und Nutzen in einem angemessenen Verhältnis stehen, oder ob die Vermeidung von Fehlern auch durch ein 4-Augen Prinzip geschehen kann.
