
\documentclass[german,11pt,a4paper]{report} % a. Set paper size to Letter, 8½ x 11. 
%\usepackage[letterpaper,margin=1in]{geometry} % e. Set margins of 1 inch (2.54 cm.) on all four sides of the paper. 
\usepackage[letterpaper,left=3cm,right=3cm, top=3.0cm, bottom=3.0cm, footnotesep=1.0cm]{geometry} 
\usepackage{mathptmx} % d. ...in a simple roman face except where indicated below (§3). 
%\usepackage[onehalfspacing]{setspace} % b. Set line spacing to 1.5 throughout the document.  
\usepackage{fancyhdr} 
\usepackage{relsize}
\usepackage{csquotes}
\usepackage{caption}
\usepackage{float}
\usepackage{wrapfig}
\usepackage{chngcntr}
\counterwithout{footnote}{chapter}

\usepackage[bottom]{footmisc}
\usepackage{tabularx}
    
\pagestyle{empty}        % No page numbers

%%%Using XeTeX (xelatex, lulatex):
\usepackage{polyglossia}
\usepackage{fontspec}
\usepackage{xunicode}
\usepackage{xltxtra}
\usepackage{url}
\usepackage{hyperref}
%\usepackage[ngerman]{babel}
%\usepackage{german}
\setdefaultlanguage[spelling=new, babelshorthands=true]{german}

\setmainfont[Mapping=tex-text]{Linux Libertine} %Falls nicht vorhanden müssen die LinLibertine-ttf-Dateien nach C:\windows\fonts verschoben werden



\usepackage{booktabs}    % For nice-looking tables
%\usepackage{natbib}      % Citation support (required for crossrefs)

%\usepackage{expex}
%\bibpunct[:]{(}{)}{;}{a}{}{,} % Defaults for in-text citations
%\usepackage{bibentry}    % Print individual references


\usepackage[style=alphabetic,backend=biber,sorting=ynt,maxnames=99,sortlocale=de-DE]{biblatex}
\addbibresource{Literatur.bib}

\usepackage{acronym}
\usepackage{multicol}
\newcommand*\elide{\textup{[\,\dots]}\xspace}

\begin{document}
\begin{huge}
\begin{center}
       \includegraphics[width=0.8cm]{./logo/logo.eps} FernUniversität in Hagen\\
				01919 Seminar Programmiersysteme
    \end{center}
\end{huge}

\vspace{3cm}

%%%Block Hausarbeit:
%\begin{center}
%\begin{large}
%Titel des Seminars\\
%Name des Seminarleiters/der Seminarleiterin
%\end{large}
%\end{center}
%%%Ende Block Hausarbeit

\title{Continuous Integration und Jenkins}
\date{\vspace{-5ex}}
{\let\newpage\relax\maketitle}
\thispagestyle{empty}


%%%Block Masterarbeit:
\begin{center}
\begin{large}
\begin{Large}
Seminararbeit\\
\end{Large}
im Studiengang 'Bachelor Informatik' \\
\end{large}
\end{center}
\begin{center}
vorgelegt von\\
\begin{large}
Daniel Wolfschmidt\\
\end{large}
\end{center}
\vspace{1cm}
\begin{center}
\begin{large}
Betreuer: Daniela Keller\\
\end{large}
\end{center}
%%%Ende Block Masterarbeit

\begin{center}
\begin{large}
Ablieferungstermin: \date{\today} \\
\end{large}
\end{center}

\vspace{1,5cm}

\begin{center}
\begin{large}
\author{Daniel Wolfschmidt}\\
\end{large}
Fließbachstraße 18\\ 
91052 Erlangen\\ 
\url{mailto:DanielWolfschmidt@gmx.de}\\
Matrikelnr.: 9601244  \\
\end{center}



\newpage
\setcounter{page}{1}
\tableofcontents
\newpage
\listoffigures
\newpage
\listoftables
\newpage


\pagestyle{fancy}
%\fancyhf{}
%\fancyhead[R]{\thepage}
%\renewcommand{\headrulewidth}{0pt} %obere Trennlinie


\fancyhf{} % clear all header and footers
\renewcommand{\headrulewidth}{0pt} % remove the header rule
\cfoot{\thepage}
\fancypagestyle{plain}{%
  \fancyhf{}%
  \fancyhf[cf]{\thepage}%
}


\section{Einleitung}
In den Jahrzehnten der Historie von Softwarentwicklung gab es immer wieder neue Erkenntnisse und neue State-of-the-Art Methoden der Entwicklung von Software. So gab es lange Zeit große, monolithische Desktopapplikationen, welche nur als großes Ganzes funktionierten. Mittlerweile geht der Trend hin zu Microservices\footnote{Dabei handelt es sich um eine Zusammenstellung unabhängiger Prozesse, die durch eine sehr leichtgewichtige Kommunikationsschicht verbunden sind. Ein Beispiel zur Kommunikation ist HTTP(S). Weitergehende Informationen z.B. unter \cite{fowler-Microservice}}. Diese, bis auf die Schnittstellenbeschreibung, unabhängige Entwicklung der einzelenen Komponenten von Software, erlaubt eine wesentlich schnellere Entwicklung von Software.\\
Auch die Einstellung der Nutzer hat sich verändert. Durch das schnelle Entwickeln von Patches und Updates, ist der Anwender zum Betatester von Software avanciert. Er ist es nicht nur gewohnt, dass in schneller Abfolge neue Änderungen an der Software veröffentlicht werden, sondern er erwartet es regelrecht.\\
Ein weiterer Aspekt sind die neuen Vorgehensmodelle im (Software-)Projektmanagement. In der Vergangenheit war es gang und gäbe, das Wasserfall Modell zu verwenden. Dabei wird der Test in einer späten Projektphase durchgeführt, und die Entwicklungsphase dauert sehr lange, bis das Gesamtprodukt fertig entwicklet ist. Heutzutage bedient man sich agiler Modelle wie z.B. Scrum. Hierbei wird in  regelmäßigen Abständen eine überschaubare Verbesserung des Produkts veröffentlicht. Dies unterstützt die oben beschriebene Veränderung in modernen Software Architekturen.\\
In diesen Zeiten immer kürzerer Entwicklungszyklen gewinnt die Entwicklung von Konzepten zur Sicherung der Code Qualität zunehmend an Bedeutung. Eines dieser Konzepte, das ich in der hier vorliegenden Seminararbeit näher beleuchten möchte, ist \textbf{Continuous Integration}.\\
Software soll schnell entwickelt und getestet werden. Dies ist nur durch eine weitreichende Automatisierung von Build-, Integrations- und Testschritten möglich. Mehrere kommerzielle und kostenlose Tools zur Unterstützung von Continuous Integration existieren am Markt, wobei diese Arbeit \textbf{Jenkins} genauer vorstellt. 
%\section{Einleitung}
%In den Jahrzehnten der Historie von Softwarentwicklung gab es immer wieder neue Erkenntnisse und neue State-of-the-Art Methoden der Entwicklung von Software. So gab es lange Zeit große, monolithische Desktopapplikationen, welche nur als großes Ganzes funktionierten. Mittlerweile geht der Trend hin zu Microservices\footnote{Dabei handelt es sich um eine Zusammenstellung unabhängiger Prozesse, die durch eine sehr leichtgewichtige Kommunikationsschicht verbunden sind. Ein Beispiel zur Kommunikation ist HTTP(S). vgl. \url{https://martinfowler.com/microservices/}}. Diese, bis auf die Schnittstellenbeschreibung, unabhängige Entwicklung der einzelenen Komponenten von Software, erlaubt eine wesentlich schnellere Entwicklung von Software.\\
%Auch die Einstellung der Nutzer hat sich verändert. Durch das schnelle Entwickeln von Patches und Updates, ist der Anwender zum Betatester von Software avanciert. Er ist es nicht nur gewohnt, dass in schneller Abfolge neue Änderungen an der Software veröffentlicht werden, sondern er erwartet es regelrecht.\\
%Ein weiterer Aspekt sind die neuen Vorgehensmodelle im (Software-)Projektmanagement. In der Vergangenheit war es gang und gäbe, das Wasserfall Modell zu verwenden. Dabei wird der Test in einer späten Projektphase durchgeführt, und die Entwicklungsphase dauert sehr lange, bis das Gesamtprodukt fertig entwicklet ist. Heutzutage bedient man sich agiler Modelle wie z.B. Scrum. Hierbei wird in  regelmäßigen Abständen eine überschaubare Verbesserung des Produkts veröffentlicht. Dies unterstützt die oben beschriebene Veränderung in modernen Software Architekturen.\\
%In diesen Zeiten immer kürzerer Entwicklungszyklen gewinnt die Entwicklung von Konzepten zur Sicherung der Code Qualität zunehmend an Bedeutung. Eines dieser Konzepte, das ich in der hier vorliegenden Seminararbeit näher beleuchten möchte, ist \textbf{Continuous Integration}.\\
%Software soll schnell entwickelt und getestet werden. Dies ist nur durch eine weitreichende Automatisierung von Build-, Integrations- und Testschritten möglich. Mehrere kommerzielle und kostenlose Tools zur Unterstützung von Continuous Integration existieren am Markt, wobei diese Arbeit \textbf{Jenkins} genauer vorstellt.


%automatisierte, toolunterstützte Integration von Softwarekomponenten zunehmend an Bedeutung. 

%Dieses Dokument dient als Vorlage für Haus- bzw. Masterarbeiten\footnote{Die entsprechend kommentierten Blöcke im Bereich der Titelseite sowie die abschließenden Seiten (Lebenslauf und Erklärung) sind bei einer Hausarbeit in der \texttt{tex-Datei} zu entfernen.} des Studiengangs 'Cultural and Cognitive Linguistics' (LMU München). Es basiert auf einem LaTeX\footnote{LaTeX kann hier heruntergeladen werden: \url{https://www.latex-project.org/}; nach der Installation kann die \texttt{tex-Datei} mit einem LaTeX-Editor, z. B. TeXShop (Mac, bereits mitinstalliert) oder Texmaker (Windows, muss zusätzlich installiert werden) geöffnet werden. Das Kompilieren (Erstellen des Pdfs) erfolgt mit XeLaTeX.}-Template, das auf der ATS-Homepage erhältlich ist (= \verb+tex-Datei+ sowie weitere Templates für die Bibliographie); das Template verwendet u. a. das \emph{ExPex}-Paket zur linguistischen Glossierung.

%Die Vorlage orientiert sich an den Richtlinien der Zeitschrift \emph{Language} (\url{http://www.linguisticsociety.org/sites/default/files/style-sheet.pdf}); dazu wird eine leicht modifizierte \verb+bst-Datei+ (\url{http://ron.artstein.org/software.html}) verwendet (s. Kapitel \ref{sec:cit}), die Bibliographie wird mit Hilfe von BibTeX generiert (s. \verb+bib-Datei+); der verwendete Font \verb+Linux Libertine+ ist Teil des Template-Pakets (die \verb+ttf-Dateien+ müssen in den \verb+font-Ordner+ des Systems kopiert werden).

\pagebreak
\chapter{Continuous Integration}
%Dieses Kapitel beschäftigt sich mit einem der beiden Hauptthemen, nämlich Continuous Integration. 
Zunächst nähert sich diese Arbeit dem Thema durch eine genaue Begriffsbestimmung, wobei auch eine Abgrenzung zu anderen, ähnlichen Begriffen eine wichtige Rolle spielt. Im weiteren Verlauf sollen dann noch der grundsätzliche Ablauf sowie die Gründe zum Einsatz dieser Methodik näher beleuchtet werden.\\
\section{Begriffsklärung und Abgrenzung zu anderen Begriffen}
Den Einstieg soll eine kurze Beschreibung von Martin Fowler bilden, er gilt als der geistige Vater von Continuous Integration und wird mit diesem Artikel in vielen anderen Abhandlungen zu dem Thema zitiert:
\begin{center}
	\textit{
	Continuous Integration is a software development practice where members of a team integrate their work frequently, usually each person integrates at least daily - leading to multiple integrations per day. Each integration is verified by an automated build (including test) to detect integration errors as quickly as possible}\\ \cite{fowler-CI}
\end{center}
Es geht hier also um das kollaborative Arbeiten in einem Team, insbesondere das Integrieren von Code in eine gemeinsame Code-Basis. Das heißt ferner, dass es eine Methodik ist, die für einen einzelnen Entwickler kaum Bedeutung hat. Das ist auch einleuchtend, denn seinen eigenen Code in eben diesen zu integrieren, geschieht automatisch durch seine Änderungen. \\
Außerdem sollte die Integration sehr oft passieren, am besten mehrmals täglich. Dies ist ein sinnvolles Vorhaben, da die Wahrscheinlichkeit auf eine sehr komplexe Integration steigt, je länger man damit wartet, denn dann wird aus dem Integrieren einer eigentlich kleinen Änderung ein umfassender Mehraufwand durch das Mergen\footnote{Mergen bezeichnet das Vergleichen mehrerer Änderungen an einer Quelldatei und das Zusammenführen ("`mergen"') dieser Änderungen. (Diese Formulierung stammt vom Autor, und ist nicht durch Literatur hinterlegt)} dieser Konflikte. Ein solcher Konflikt könnte zum Beispiel eine Änderung sein, die dieselbe Code Stelle auf zwei Branches unterschiedlich geändert hat.\footnote{Dieser Konflikt tritt im Alltag des Autors als Build- und Configuration Manager des öfteren auf und ist hier nicht durch Literatur hinterlegt} \\
Der dritte Teil der vorliegenden Beschreibung geht darauf ein, wie man den Nachweis erbringen kann, dass die Integration erfolgreich war. Dafür soll es automatisierte (und dadurch auch standardisierte) Builds geben. Diese Builds erzeugen zunächst aus dem menschenlesbaren Code den von Maschinen ausführbaren Code, sowie dazugehörige Tests in verschiedenem Detailgrad. Martin Fowler gibt in seiner Beschreibung auch an, dass die Tests mit zu diesen automatisierten Builds gehören. Hierbei muss man auf eine sinnvolle Testtiefe achten. Wenn man die Test-Pyramide\footnote{Dabei handelt es sich um eine Darstellung der unterschiedlichen Testtypen in hierarchischer Form, wobei von unten nach oben die Geschwindigkeit abnimmt und die Kosten zunehmen. Vgl.:\cite{fowler-Testpyramid}} zu Rate zieht, gibt es neben UnitTests auch solche, die das Zusammenspiel mehrerer Komponenten testen. Dies muss jedoch durch höheren zeitlichen Aufwand erkauft werden. Deshalb muss darauf geachtet werden, dass alle Tests fertig sind, bevor der nächste Entwickler sein Änderungen in die Code-Basis integriert. Es kann vorteilhaft sein, wenn man während des Check-In Builds nur Unittests ausführt, und in einem möglicherweise nur einmal am Tag laufenden Build dann auch umfangreichere Test wie Komponententests, Integrationstests oder Tests auf der installierten Applikation, sogenannte Smoketests, ausführt.\\
Um diese Beschreibung der Kernpunkte von Continuous Integration auf eine breitere Basis zu stellen, soll hier noch eine zweite Quelle genutzt werden, um von einem anderen Blickwinkel auf das Thema zu blicken.
\begin{center}
	\textit{
The practice of continuous integration represents a fundamental shift\\ in the process of building software. It takes integration, commonly\\
an infrequent and painful exercise, and makes it a simple, core part\\ of a developer’s daily activities. Integrating continuously makes\\ integration a part of the natural rhythm of coding, an integral part\\ of the test-code-refactor cycle. Continuous integration is about\\ progressing steadily forward by taking small steps.}\\ \cite{10.1007/978-3-540-24853-8_8}
\end{center}
Der Autor dieses Konferenzbeitrags ist R. Owen Rogers. Er arbeitete zu dieser Zeit bei Thoughtworks, derselben Firma, bei der auch Martin Fowler arbeitet. Das Zitat stammt von einer Konferenz aus dem Jahr 2004, also zeitlich zwischen der initialen Version des Artikels über Continuous Integration (CI) von Martin Fowler und seiner aktuellen Version aus dem Jahr 2006.\\
Es wird dabei der Fokus eher auf die Auswirkungen von Continuous Integration auf die Softwareentwicklung und den Einfluss auf die Qualität von Software gelegt. Rogers geht vor allem darauf ein, dass das häufige Integrieren der zentrale Teil dieses Konzepts ist. Das deckt sich mit der oben vorgestellten Sichtweise von Martin Fowler. Des weiteren setzt er den Ansatz in den Kontext von "`test-code-refactor"', und geht damit auch auf einen anderen bereits vorgestellten Aspekt ein, nämlich das Überprüfen des Erfolgs der Integration. 
Hier wird im Gegensatz zu Fowler nicht explizit auf den Team-Aspekt eingegangen. Dieser Gesichtspunkt ist eher implizit enthalten.\\
Die Sichtweise von R. Owen Rogers deckt sich damit mit der von Martin Fowler, er beleuchtet das Thema einfach nur aus einem anderen, eher anwendungsbezogenen Blickwinkel. Das ist auch nachvollziehbar, da es sich hier nicht um eine theoretische Abhandlung handelt, sondern einen Konferenzbeitrag, der an Anwender dieser Technik gerichtet war.\\

Zusammenfassend bleibt zu sagen, dass mit Continuous Integration die Zusammenarbeit eines Entwicklerteams an einer gemeinsamen Code-Basis verbessert werden soll. Dies soll durch kontinuierliches Zusammenführen der Änderungen aller Beteiligten und das automatisierte Prüfen des Ergebnisses geschehen.
\subsection*{Abgrenzung zu anderen Begriffen}
Es gibt einige Begriffe, die Continuous Integration sehr ähnlich sind. Im Folgenden soll genauer umrissen werden, worin der Unterschied und eventuelle Gemeinsamkeiten liegen.
\begin{itemize}
	\item\textbf{Continuous Delivery}\label{Continuous Delivery}\\
	Eine zusammenfassende Definition von Martin Fowler:
\begin{center}
	\textit{
	You achieve continuous delivery by continuously integrating the \\software done by the development team, building executables, and\\ running automated tests on those executables to detect problems. \\Furthermore you push the executables into increasingly production-like \\environments to ensure the software will work in production. To do \\this you use a DeploymentPipeline.}\\ \cite{fowler-CD}
\end{center}
	Bei Continuous Delivery (CD) wird der Gedanke von Continuous Integration aufgegriffen, und weiterentwickelt. Während Continuous Integration sich komplett in der Entwicklung bewegt, umfasst Continuous Delivery auch Schritte bis hin zum Kunden. Es werden weitere Schritte wie das Paketieren als Deliverable (z.B. Erstellen eines Setups) und das Deployment (z.B. Bereitstellen als Download oder Einstellen in einen AppStore) betrachtet. Das Ziel dessen ist, dass das Ergebnis der Entwicklung zum geplanten Auslieferungszeitpunkt bereit ist für den Kunden.\\
	\item\textbf{Continuous Deployment}\label{Continuous Deployment}\\
	Hier beziehe ich mich auf den Abstract eines Konferenzbeitrages von Helena Holmström Olsson
	\begin{center}
		\textit{
		The concept of continuous deployment, i.e. the ability to deliver software functionality frequently to customers \textelp{}}\\
		\cite{olsson2012climbing}
	\end{center}
	Continuous Deployment bringt das CI-Konzept noch einen Schritt weiter als Continuous Delivery. Nicht nur wird hier wie bei Continuous Delivery in "`Production-like"' Umgebungen installiert, sondern sehr häufig (potentiell mit jedem Build) in die Produktion gegeben. Der Unterschied zu Continuous Delivery wirkt zunächst nur marginal, aber prinzipiell kann man sagen, dass bei Continuous Delivery festgelegt wird, wann man in die Produktion geht, und bei Continuous Deployment dies laufend passiert. \cite{scrum-overview-ci-cd}. Das heißt aber auch, dass dieses Konzept das am weitesten automatisierte ist, mit allen Vor- und Nachteilen.
	
	
\end{itemize}

In \autoref{fig:ci-cd-image-1024x384} sind die vorgestellten Begriffe in Kontext zueinander gesetzt, wobei zusätzlich noch Agile Development und DevOps genannt sind, die aber hier nicht näher erläutert werden.

\begin{figure}[h]
  \centering
  \fbox{\includegraphics[width=\textwidth]{./Images/ci-cd-image-1024x384.png}}
  \caption{Einordnung von CI und ähnlicher Begriffe \cite{Begriff-Overview}}\label{fig:ci-cd-image-1024x384}
\end{figure}

\section{Ablauf von Continuous Integration}
In diesem Kapitel beziehe ich mich auf den Ablauf von CI wie er im Unterabschnitt "`Building a Feature with Continuous Integration"' von \cite{fowler-CI} beschrieben wird.
Da in der Beschreibung von CI die Rede von der Integration von Codeänderungen ist, muss es auch eine gemeinsame Basis geben, in die diese Änderungen einfließen. Eine solche gemeinsame Basis ist ein sogenanntes Source-Control-Management-System (SCM). Dabei handelt es sich um ein System, in dem Änderungen an einer Datei, oder der Struktur der Dateien, nachvollziehbar gemacht werden, und man verschiedene Stände abrufen kann (vgl. \cite{fowler-CI}). 
Es löst auch viele andere Probleme, die bei der Zusammenarbeit entstehen können, wie das gegenseitige Überschreiben von Änderungen. \\
Verschiedene Konzepte existieren hierzu, die entweder eine zentrale Stelle, an der alle Dateien inklusive Historie verwaltet werden, haben, oder verteilte Systeme, bei der es keine ausgezeichnete zentrale Instanz gibt.\\
In \autoref{fig:Schema-aufbau} ist der schematische Ablauf von CI zu sehen, wobei auf jeden der Schritte im Folgenden genauer eingegangen wird. Bidirektionale Pfeile stellen dabei eine Aktion dar, die einer Art Request-Response entsprechen, wie z.B. Dateien aus dem SCM holen entspricht Aktion starten und in der Rückrichtung Dateien erhalten.
\begin{figure}[h]
  \centering
  \fbox{\includegraphics[width=\textwidth]{./Images/Schema-aufbau.png}}
  \caption{Schematischer Ablauf von CI}\label{fig:Schema-aufbau}
\end{figure}
\begin{enumerate}
	\item \textbf{Quellen holen}\\% 1
		Die Arbeit des Entwicklers basiert auf dem aktuellen Stand der Quellen aus dem SCM. Deshalb holt er sich zunächst diesen auf seinen lokalen Computer, um seine Arbeit zu beginnen.
		\item \textbf{Änderungen implementieren}\\% 2
		Der Entwickler folgt diesem Prozess aus einem bestimmten Grund, nämlich entweder um ein neues Feature zu implementieren oder bekannte Fehler in der Software zu beheben. Denkbar ist auch, dass er hier fehlende Tests nachliefert oder fehlschlagende Tests in Ordnung bringt. Dies geschieht in diesem Schritt. Der Entwickler führt die ihm übertragenen Aufgaben aus. Dabei wird bei CI besonderer Wert auf Tests gelegt. Dieser sehr hohe Stellenwert von Tests ist mittlerweile zum Quasi-Standard in der Softwareentwicklung geworden und folgt dem schematischen Aufbau der Testpyramide, wie sie in (\cite{fowler-Testpyramid}) beschrieben wird.
		\item \textbf{Lokal bauen und testen}\\% 3
		Der vorhergehende Schritt hat sich komplett auf das Implementieren und Ändern von Code und Tests beschränkt. Diese müssen auch noch auf Fehlerfreiheit überprüft werden, bevor man sie in das SCM einfügt. Die erste Kontrollinstanz ist nun der automatisierte Build auf der Entwicklermaschine. Darunter versteht man das Kompilieren und Linken\footnote{Dabei werden einzelne Objekt-Dateien zu Bibliotheken oder ausführbaren Dateien zusammengeführt} der Quellen, sowie das Ausführen der zuvor geschriebenen automatisierten Tests. Hierbei beschränkt man sich zumeist auf UnitTests.
		\item \textbf{Delta der Quellen holen}\\% 4
		Während ein Entwickler seinen Auftrag ausgeführt hat, könnten einige seiner Kollegen bereits ihre Arbeit vollendet haben. Deshalb sollte er nun den aktuellen ("`top-level"') Stand holen, weil es sonst zum Beispiel passieren könnte, dass Änderungen überschrieben werden oder Konflikte beim Hinzufügen zum SCM entstehen, die nicht automatisch aufgelöst werden können.
		\item \textbf{Konflikte auflösen}\\% 5
		Falls der Entwickler nun in die Situation gekommen ist, dass seine lokalen Änderungen mit Änderungen, die bereits im SCM sind, in Konflikt stehen, so muss er diese manuell auflösen. Es gibt einige Konflikte, die grundsätzlich auch toolunterstützt automatisch auflösbar sind (z.B. Änderung an derselben Datei, aber in unterschiedlichen Zeilen), jedoch können manche Konflikte nur durch manuellen Eingriff eines Menschen aufgelöst werden. Dies gilt vor allem, wenn der semantische Zusammenhang der Änderungen eine wichtige Rolle spielt.
		\item \textbf{Lokal bauen und testen}\\% 6
		Der Entwickler muss durch nochmaliges Bauen überprüfen, ob seine Änderungen noch funktionieren und alle Tests weiterhin fehlerfrei sind. Dies betrifft nun nicht nur seinen eigenen Code, sondern auch den der anderen Entwickler. Es ist denkbar, dass seine Änderungen Auswirkungen an den Code-Stellen oder Tests hat, die in der Zwischenzeit entstanden sind. Dabei ist er selbst in der Verantwortung, dass der Code-Stand, den er später dem SCM hinzufügen möchte, funktioniert. 
		\item \textbf{Quellen einchecken mit bereits aufgelösten Konflikten}\\% 7
		Nachdem lokal der Code erfolgreich kompilierbar ist, sowie alle (Unit-)Tests fehlerfrei sind, kann der Entwickler seine Änderungen dem SCM hinzufügen.
		\item \textbf{Änderungen starten neuen Build}\\% 8
		Je nach SCM und Buildserver-Kombination gibt es mehrere denkbare Ansätze automatisiert nach jedem Check-In von Code einen Build zu triggern. Diese basieren sehr häufig auf dem Observer-Pattern\footnote{Ziel des Observer Patterns ist es eine sog. one-to-many Beziehung zwischen Objekten zu definieren, so dass eine Statusänderung eines Objektes all davon abhängigen Objekte benachrichtigt, bzw. automatisch ändert. vgl. auch \cite{hannemann2002design}} entweder mit Push-Notification (Alle Subscriber werden bei einer Änderung benachrichtigt) oder Pull-Notification (Regelmäßiges Nachfragen durch die Subscriber, ob sich etwas geändert hat). Egal welche der Methoden zum Einsatz kommt, nach dem Hinzufügen der Änderungen wird ein Build gestartet.
		\item \textbf{Quellen aus dem SCM System holen}\\% 9
		Der Buildserver wurde bisher nur benachrichtigt, es ist aber noch kein Inhalt übertragen worden. Das geschieht in diesem Schritt. Der Einfachheit halber wurde in \autoref{fig:Schema-aufbau} nur ein einziger Buildserver eingefügt. Prinzipiell gibt es sehr häufig eine Orchestrierungs-Instanz und mehrere Buildserver. Diese zentrale Instanz koordiniert dabei die Aufgaben und die Server führen diese aus. Im vorliegenden Schaubild ist sowohl die koordinierende Instanz als auch die ausführende auf demselben Server. Die Übertragung der Quellen findet auf die ausführende Instanz statt, da diese auch den Code kompiliert und testet.
		\item \textbf{Quellen bauen und testen auf dem Server}\\% 10
		Anschließend wird, genau wie auch auf dem lokalen Rechner des Entwicklers, der Code gebaut und getestet. Der große Vorteil gegenüber den Entwicklerrechnern ist, dass es sich um eine klar definierte Instanz handelt. Entwicklerrechner sind sehr heterogen vom Aufbau, da im Laufe der Zeit immer mehr "`HilfsTools"', die das Arbeiten erleichtern, oder Bibliotheken hinzukommen. Der Buildserver ist anders, denn er verfügt über eine genau festgelegte Zusammenstellung von installierten Programmen und Bibliotheken, die zum Erstellen der Kompilate und dem Testen, benutzt werden können. Hier fällt zum ersten Mal auf, falls der Entwickler sich auf etwas verlässt, das nur auf seiner Maschine vorhanden ist. Dazu zählen zum Beispiel neue Kompilate. Wenn die Source Dateien nicht explizit in das SCM eingefügt wurden bzw. die Anweisungen zum Erstellen des Kompilats aus diesen Dateien fehlen, funktioniert zwar der lokale Build, aber nicht der Server Build. Dies ist das zentrale Quality Gate im CI Prozess und aufgrund des automatisierten Prozesses und der klar definierten Umgebung die Komponente, in deren Ergebnis das größte Vertrauen zu setzen ist.
		\item \textbf{Rückmeldung über das Ergebnis des Build}\\% 11
		Nachdem der Build fertig ist, muss der Entwickler auch noch auf irgendeine Art und Weise Kenntnis vom Ergebnis erlangen. Entweder wird es auf einer Webseite veröffentlicht, oder er bekommt direkt eine Benachrichtigung mit dem Ergebnis, oder eine Kombination aus beidem. Dies ist wichtig, denn abhängig vom Ergebnis des Builds hat dies Auswirkungen auf die weitere Arbeit. Der schlechtere Fall ist, dass der Build nicht funktioniert hat. Das bedeutet, dass alle anderen Entwickler, die sich auf diesen Build stützen müssen, blockiert sind in ihrer Arbeit. Dies hat zur Folge, dass so schnell wie möglich der Grund gefunden werden muss und dieses Problem behoben wird. Potentielle Lösung wäre auch, die Änderungen im SCM rückgängig zu machen, damit die anderen Entwickler vorerst ungestört weiter arbeiten können. Im guten Fall, in dem der Build erfolgreich war, signalisiert die Benachrichtigung, dass der Entwickler sich nun der nächsten Aufgabe widmen kann.
\end{enumerate}
%Ein Entwickler nimmt sich nun den aktuellen Stand aus diesem SCM, und führt seine Arbeit aus. Dabei wird bei CI besonders Wert auf Tests gelegt. Dies ist mittlerweile zum Quasistandard in der Sotwareentwicklung geworden, und folgt vom schematischen Aufbau her der Testpyramide, wie sie in (\cite{fowler-Testpyramid}) beschrieben wird.\\
%Die erste Kontrollinstanz ist nun der automatisierte Build auf der Entwicklermaschine. Darunter versteht man das Kompilieren und Linken der Quellen sowie das Ausführen der zuvor geschriebenen automatisierten Tests. Hierbei beschränkt man sich zumeist auf UnitTests.\\
%Wenn dieser Schritt zufriedenstellend abgeschlossen ist, können die Änderungen zurück in das SCM geführt werden. Dazu holt der Entwickler sich abermals den aktuellen Stand, und löst potentielle Konflikte auf, die entstanden sind, weil andere Entwickler in der Zwischenzeit ihren Stand eingefügt haben. Lokal hat er nun bereits den neuen Stand, den er nun noch einmal lokal baut und testet. Ist dieser lokale Build erfolgreich, fügt er seine Änderungen dem SCM hinzu. An diesem Punkt startet der automatische Prozess der Buildumgebung. Sobald die Änderungen im Source Control sind, wird ein Build getriggert. Der Einfachheit halber habe ich dabei in Abbildung~\ref{fig:Schema-aufbau} nur einen einzigen Build Server eingefügt. Prinzipiell gibt es eine Orchestrierungs-Instanz und mehrere Build Server. Diese zentrale Instanz koordiniert dabei die Aufgaben und die Server führen diese aus.
\section{Gründe Continuous Integration einzusetzen}\label{sec:Gründe Continuous Integration einzusetzen}
Die Gründe für den Einsatz von Continuous Integration sind sehr vielfältig, daher im Folgenden eine kleine Auswahl an Gründen für CI\footnote{Diese Auswahl an Gründen basiert auf Gesprächen des Autors auf Konferenzen zum Thema CI/CD/DevOps sowie der persönlichen Erfahrungen des Autors}:
\begin{enumerate}
	\item \textbf{Qualität steigern}\\
	Dieser Grund ist ziemlich offensichtlich. Durch die regelmäßig im Build mitlaufenden Tests, die ein schnelles Feedback über die Qualität des Codes geben, wird diese auf lange Sicht gesteigert. Es wäre auch denkbar gewisse Metriken einzuführen, die einen Build scheitern lassen, so dass die Entwickler gezwungen sind die Qualität zu erhöhen. Darunter zählt beispielsweise Code Coverage. Dabei geht es darum, wie viel des produktiven Codes von Tests durchlaufen wird und dass dadurch eine Qualitätsaussage darüber getroffen werden kann.	
	\item \textbf{Audit Trail}\\
	Die Einführung von Praktiken wie Continuous Integration und deren Implementierung als ganzes System helfen immens bei der Softwareentwicklung in regulatorischen Umgebungen. Besonders die FDA\footnote{Food and Drug Administration, eine US Amerikaische Behörde ähnlich dem deutschen Gesundheitsministeriums. Bei der Entwicklung von Software im Kontext von Medizinprodukten macht die FDA strikte Vorgaben zur Nachvollziehbarkeit und Dokumentation} macht strikte Vorgaben in Bezug auf die Nachvollziehbarkeit von Änderungen, bzw. dem Einfluss den diese auf ein Produkt haben ("`Audit Trail"').
	\item \textbf{Schnellerer und spontanerer Release möglich}\\
	Unter Zuhilfenahme von CI wird der Prozess der Softwareentwicklung weiter voran getrieben als in einem klassischen Setup. Es wird mit jeder Codeänderung getestet und auch die Komponenten untereinander integriert. Das verkürzt den Restprozess bis zum Release und steigert auch das Vertrauen in den aktuellen Stand, da dieser regelmäßig und mehrfach getestet ist. Wenn aus dem Markt nun ein besonders schwerwiegender Fehler gemeldet wird, kann man kurzfristig einen (Patch-)Release ansetzen und durchführen.
	\item \textbf{Vorsichtigere Entwickler, wenn sie wissen, dass eine Kontrollinstanz existiert}\\
	Auch die Einstellung der Entwickler ändert sich. Alleine durch das Wissen, dass es eine zentrale Kontrollinstanz gibt, gehen sie bewusster und vorsichtiger mit Codeänderungen um. Jeder im System kann sehen, aufgrund welches Check-Ins der Build auf einmal nicht mehr funktioniert. Schon allein weil man vor den Kollegen nicht als schlechter Entwickler identifiziert werden möchte, achtet man mehr auf seine Check-Ins, baut und testet lokal. Aufgrund dieses vorsichtigeren Ansatzes werden auch die Check-Ins vom Umfang her kleiner. Das System ist automatisiert und es macht von dieser Seite keinen Unterschied, ob man viel oder wenig ändert. Kleine Änderungen lassen sich jedoch leichter korrigieren bei einem Fehler und auch leichter kontrollieren im Handling. Das führt insgesamt zu einer besseren Entwicklungskultur im Unternehmen und zu besserer Performance der Mitarbeiter.
	\item \textbf{Management-Vorgaben}\\
	Auch dieser, eher organisatorische, Grund ist vorzubringen. Dadurch, dass Continuous Integration, bzw. dessen Weiterentwicklung CD, mittlerweile Einzug gehalten hat in weite Teile der Softwareentwicklung, kann auch das Management verlangen, dass dies eingeführt wird, bzw. es als Abteilungs- oder Unternehmensziel festlegen. Es sollte jedoch sowieso im eigenen Interesse der Softwareentwicklung sein, solche Praktiken anzuwenden.
\end{enumerate}
	
\section{Mögliche Verbesserungen}
Das Konzept CI wurde bereits 2006 von Martin Fowler vorgestellt. Seitdem gab es bereits mehrere Ansätze der Weiterentwicklung. Ich möchte hierbei einige vorstellen, sowohl auf Seiten des Prozesses als auch solche, die durch neue Funktionen von Tools ermöglicht wurden:
\begin{itemize}
	\item \textbf{Erweiterung des Prozesses}\\
 Aufgrund der immer kürzeren Entwicklungszyklen wird es in manchen Bereichen, wie z.B. App Entwicklung für mobile Geräte, unausweichlich möglichst viel der Arbeit zu automatisieren. Deshalb wurde bereits die Erweiterung des Konzepts zu "`Continuous Delivery"'(\autoref{Continuous Delivery}) bzw. "`Continuous Deployment"'(\autoref{Continuous Deployment}) entwickelt. Hierfür gibt es auch Toolunterstützung von z.B. Jenkins.
	\item \textbf{Verbesserung des vorhandenen Prozesses}\\
Eine weitere mögliche Verbesserung setzt viel früher im Prozess an. In dem hier vorgestellten klassischen CI Prozess fügt ein Entwickler seine Änderungen in das SCM System ein und anschließend wird die Qualität durch einen automatisierten Build ermittelt. Das kann aber gerade im Fall eines mangelhaften Check-Ins zu Problemen führen, da andere Entwickler diesen korrupten Stand aus dem SCM holen und eventuell in ihre Änderungen einbauen.\\
Deshalb gibt es bereits vorhandene Konzepte die Änderungen zu prüfen, bevor diese in das SCM gelangen. Je nach verwendetem Tooling heißen diese "`Gated-Checkin"'\cite{TFS-GC} (TFS\footnote{Microsoft Team Foundation Server, kommerzielles Tool zur Unterstützung von CI}) oder "`Pre-tested Commits"'\cite{Jenkins-GC} (Jenkins).

\end{itemize} 

%\section{Continuous Integration}
%Dieses Kapitel beschäftigt sich mit einem der beiden Hauptthemen, nämlich Continuous Integration. Dabei möchte ich mich diesem Thema zunächst durch eine genaue Begriffsbestimmung nähern, wobei auch eine Abgrenzung zu anderen, ähnlichen Begriffen eine wichtige Rolle spielt. Im weiteren Verlauf sollen dann noch der Einsatz von Tools zur Unterstützung sowie die Gründe zum Einsatz dieser Methodik näher beleuchtet werden.\\
%\subsection{Begriffsklärung}
%Den Einstieg soll eine kurze Beschreibung von Martin Fowler bilden, er gilt als der gesitige Vater von Continuous Integration und wird mit diesem Artikel in vielen anderen Abhandlungen zu dem Thema zitiert:
%\begin{center}
%	\textit{
%	Continuous Integration is a software development practice where members of a team integrate their work frequently, usually each person integrates at least daily - leading to multiple integrations per day. Each integration is verified by an automated build (including test) to detect integration errors as quickly as possible} \cite{fowler-CI}
%\end{center}
%Es geht hier also um das kollaborative Arbeiten in einem Team, insbesondere das Integrieren von Code in eine gemeinsame Code-Basis. Das heißt ferner, dass es eine Methodik ist die für einen einzelnen Entwickler kaum Bedeutung hat. Das ist auch einleuchtend, denn seinen eigenen Code in eben diesen zu integrieren, macht kaum Sinn. \\
%Ferner sollte dies sehr oft passieren, am besten mehrmals täglich. Auch das ist eine sehr sinnvolle Daumenregel. Wenn man zu lange seinen Code nicht in die gemeinsame Code-Basis integriert, so besteht die Gefahr, dass man zu weit weg ist davon und dadurch immense Aufwände entstehen. Dann wird aus dem Integrieren einer eigentlich keinen Änderung eine umfassende Merge\footnote{Mergen bezeichnet das Vergleichen mehrerer Änderungen an einer Quelldatei und das Zusammenführen ("`mergen"') dieser Änderungen. vgl.:\url{https://de.wikipedia.org/wiki/Merge}}-Session.\\
%Der dritte Teil der vorliegenden Beschreibung geht darauf ein, wie man den Nachweis erbringen kann, dass die Integration erfolgreich war. Dafür soll es automatisierte (und dadurch auch standardisierte) Builds geben. Diese Builds erzeugen zunächst aus dem menschenlesbaren Code den von Maschinen ausführbaren Code sowie dazu gehörige Tests in verschiedenem Detailgrad. Martin Fowler gibt in seienr Beschreibung auch an, dass die Tests mit zu diesem automatisierten Builds gehören. Hierbei muss man auf eine sinnvolle Testtiefe achten. Wenn man die Test-Pyramide\footnote{Dabei handelt es sich um eine Darstellung der unterschiedlichen Testtypen in hierarchischer Form, wobei von unten nach oben die Geschwindigkeit abnimmt und die Kosten zunehmen. vgl.:\url{https://martinfowler.com/bliki/TestPyramid.html}} zu Rate zieht, muss man darauf achten, dass alle Tests fertig sind bevor der näcshte Entwickler sein Änderungen in die Code-Basis integriert. Deshalb ist es eventuell am Besten, wenn man während des initialen Builds nur Unittests ausführt, und ein möglicherweise nur einmal am Tag laufender Build dann detailiierter testet.\\
%Um diese Beschreibung der Kernpunkte von Continuous Integration auf eine breitere Basis zu stellen, möchte ich noch eine zweite Quelle nutzen, um von einem anderen Blickwinkel auf das Thema zu blicken.
%\begin{center}
%	\textit{
%The practice of continuous integration represents a fundamental shift\\ in the process of building software. It takes integration, commonly\\
%an infrequent and painful exercise, and makes it a simple, core part\\ of a developer’s daily activities. Integrating continuously makes\\ integration a part of the natural rhythm of coding, an integral part\\ of the test-code-refactor cycle. Continuous integration is about\\ progressing steadily forward by taking small steps.}\\ \cite{10.1007/978-3-540-24853-8_8}
%\end{center}
%Der Autor dieses Konferenzbeitrags ist R. Owen Rogers. Er arbeitet bei Thoughtworks, derselben Firma bei der auch Martin Fowler arbeitet. Er stammt von einer Konferenz aus dem Jahr 2004, also zeitlich zwischen der initialen Version über CI\footnote{Ab hier werde ich CI als Abkürzung für "`Continuous Integration"' verwenden. Diese Abkürzung ist auch im Abkürzungsverzeichnis zu finden.} seiner aktuellen Version aus dem Jahr 2006.\\
%Es wird dabei eher der Fokus auf die Auswirkungen von Continuous Integration auf die Softwareentwicklung und der Einfluss auf die Qualität von Software gelegt. Er geht dabei vor allem darauf ein, dass das häufige Integrieren der zentrale Teile dieses Konzepts ist. Das deckt sich mit der oben vorgestellten Sichtweise von Martin Fowler. Desweiteren setz er den Ansatz in den Kontext von "`test-code-refactor"', und geht damit auch auf einen anderen bereits vorgestellten Aspekt ein, nämlich das Überprüfen des Erfolgs der Integration. Der Anteil, dass es eine Praktik ist, die hauptsächlich in Teams sinn macht, ist eher implizit enthalten in diesem Text und wird nicht extra erwähnt.\\
%Er deckt sich damit mit der Sichtweise von Martin Fowler, und beleuchtet das Thema einfach nur aus einem anderen, eher anwendungsbezogenen Blickwinkel. Das ist auch nachvollziehbar, da es sich hier nicht um eine theoretische Abhandlung handelt sondern einen Konferenzbeitrag, der an Anwender dieser Technik gerichtet war.\\

%Zusammenfassend bleibt zu sagen dass mit Continuous Integration die Zusammenarbeit eines Entwicklerteams an einer gemeinsamen Code-Basis verbessert werden soll. Dies soll geschehen durch kontinuierliches Zusammenführen der Änderungen aller Beteiligten und das automatisierte Prüfen des Ergebnisses.
%\subsubsection*{Abgrenzung zu anderen Begriffen}
%Es gibt einige Begriffe die Continuous Integration sehr ähnlich sind. Dieser Abschnitt soll genauer umreißen, wo der Unterschied ist und was sie vielleicht gemeinsam haben.
%\begin{itemize}
%	\item\textbf{Continuous Delivery}\\
%	Auch hier hat Martin Fowler eine zusammenfassende Definition geliefert:
%\begin{center}
%	\textit{
%	You achieve continuous delivery by continuously integrating the \\software done by the development team, building executables, and\\ running automated tests on those executables to detect problems. \\Furthermore you push the executables into increasingly production-like \\environments to ensure the software will work in production. To do \\this you use a DeploymentPipeline.}\\ \cite{fowler-CD}
%\end{center}
%	Bei Continuous Delivery wird der Gedanke von Continuous Integration aufgegriffen, und weiterentwickelt. Während Continuous %Integration sich komplett in der Entwicklung bewegt, umfasst Continuous Delivery auch Schritte bis hin zum Kunden. Es werden weitere Schritte wie das Paketieren als Deliverable (z.B. Erstellen eines Setups) und das Deployment (z.B. Bereitstellen als Download oder Einstellen in einen AppStore) betrachtet. Das Ziel dessen ist, dass das Ergebnis der Entwicklung zu jedem Zeitpunkt zum Kunden geschickt werden könnte.\\
	
	
%\end{itemize}
%\subsection{Ablauf von Continuous Integration}
%Da in der Beschreibung von CI die Rede ist vom Integrieren von Code Änderungen, muss es auch irgendwie eine gemeinsame Basis geben, in die diese Änderungen einfließen. Eine solche gemeinsame Basis ist ein sogenanntes Source-Control-Management-System (SCM\footnote{Ab hier werde ich SCM als Abkürzung für "`Source Control Management"' verwenden. Diese Abkürzung ist auch im Abkürzungsverzeichnis zu finden.}). Dabei handelt es sich um ein System, in dem Änderungen an einer Datei nachvollziehbar gemacht werden, und man verschiedene Stände dieser Datei abrufen kann. Es gibt dazu verschiedene Konzepte, entweder eine zentrale Stelle an der alle Dateien inklusive Historie verwaltet werden, oder verteilte Systeme, bei der es keine ausgezeichnete zentrale Instanz gibt.
%\subsubsection{Beteiligte Entitäten und schematischer Ablauf}
%Ein exemplarischer Ablauf inklusive beteiligter Komponenten ist in Abbildung XY zu sehen.
%\subsection{Gründe Continuous Integration einzusetzen}
%\subsection{Marktübersicht an Tools}
%\subsubsection{Kommerziell vs. Kostenlos}
%\subsubsection{Hosted vs. On-Premise}
\pagebreak

\chapter{Tools zur Unterstützung von CI}
Es gibt eine sehr große Menge an angebotenen Tools am Markt, um CI zu unterstützen. In diesem Kapitel möchte ich vor allem ein paar dieser Tools kurz vorstellen und anhand einiger Merkmale kategorisieren.
\section{Kommerziell vs. Kostenlos}
Grundsätzlich kann man zwischen kostenlosen (meist Open Source) und proprietären, kommerziellen Tools unterscheiden. Es gibt sowohl Gründe für die eine wie auch die andere Art. Ich möchte hier einige Aspekte aufführen, welche eine Toolentscheidung in die eine oder andere Richtung lenken:
\begin{itemize}
	\item \textbf{Technologie}\\
	Zunächst ist zu erwähnen, dass bestimmte Rahmenbedingungen für Unternehmen beziehungsweise Projekte vorgegeben sind. Als Beispiele möchte ich hier bestimmte zu verwendende Technologien erwähnen. Es kann unumgänglich sein ein bestimmtes kommerzielles Produkt zu verwenden, weil nur für diese Technologie zertifizierte Tools erlaubt sind, was meist mit hohem finanziellem Aufwand verbunden ist. Kostenlose Projekte können dies nicht für jede Version leisten und lassen sich deshalb nicht zertifizieren.
	\item \textbf{Usability}\\
	Funktional sind kostenlose Tools meist sehr weit entwickelt. Sie sind nicht selten sogar funktional besser als kommerzielle Tools, weil die Entwicklergemeinde selbst die Funktionalität nach vorne treibt und sogar neueste Entwicklungen einfließen. Ein Manko in vielen dieser Tools ist jedoch die Usability. Nur die größten der kostenlosen Tools schaffen es aufgrund des riesigen Entwicklerhintergrunds, auch die Usability im Auge zu behalten. Ein Beispiel ist Jenkins, bei dem lange Zeit UE/UX das größte Manko war und erst seit Einführung von "`Blue Ocean"' 2016 ist auch der Usability Aspekt in diesem sehr großen Projekt präsent.
	\item \textbf{Regulatorische Einflüsse}\\
	Ein weiterer Aspekt, der Aufmerksamkeit bedarf sind regulatorische Rahmenbedingungen. 
	\item \textbf{Rechtliche Bedingungen}\\
	In der Softwareentwicklung gerade in sehr großen Unternehmen, sind die Kosten für ein System eher zu vernachlässigen. Viel gewichtiger sind rechtliche Absicherungen die man sich durch den Einsatz eines Tools eines finanzkräftigen Unternehmens einkauft. Falls ein Fehler auf einen Bug des CI Systems zurückzuführen ist, und das Unternehmen mit einem prozentualen Anteil des hohen Jahresumsatzes haften muss, so besteht prinzipiell die Chance etwas zurückzuholen. Bei einem kostenlosen Tool existiert dieser finanuzielle Hintergrund nicht. Deshalb ist eine Tendenz zu erkennen, je größer das Unternehmen ist, desto wahrscheinlicher ist der Einsatz eines kommerziellen Tools.
	\end{itemize}
\section{Hosted vs. On-Premise}
Auch dabei, ob man sich selbst um die Infrastruktur des Servers kümmert, oder ein fertiges Produkt mietet, kann man unterscheiden. Dabei gibt es mehrere Aspekte, die beachtet werden müssen, wobei ich einige herausgreifen möchte:
\begin{itemize}
	\item \textbf{Finanzielle Aspekte}\\
	Von finanzieller Seite, muss man klar unterscheiden. Eigene Infrastuktur kostet erstmal Anschaffungskosten, verursacht dann aber auch während des Betriebs laufende Kosten wie Strom, Wartung und Arbeitszeit (was indirekt auch Kosten sind). Auf der anderen Seite bekommt man bei On-Premise Angeboten alles aus einer Hand, trägt kein finanzielles Risiko für Hardwareausfall, und muss dafür einen monatlichen Beitrag zahlen. Für eher kleine Projekte ist es wohl eher von Vorteil On-Premise zu verwenden, in großen Unternehmen und Projekten, wo eine Person sich nur um Build Automatisierung kümmern kann, ist dann wohl eher der eigene Server besser.
	\item \textbf{Nachvollziehbarkeit und Reproduzierbarkeit}\\
	Ein anderer Aspekt ist vor allem für regulatorische Umgebungen wichtig. Hierbei muss es möglich sein, nachzuvollziehen was alles Auswirkungen auf das Ergebnis des Builds hatte. Dazu gehört auch die Build Umgebung selbst. Diese ist jedoch bei vielen On-Premise Angeboten nicht eindeutig zu identifizieren, da diese neue Features einfach live schalten. Auch ist es unmöglich einen Build mit einem bestimmten Stand der Build Automatisierung nochmals laufen zu lassen nach längerer Zeit. Man hat keine Chance, den alten Build Server wiederherzustellen.
	\item \textbf{Datenschutz und Datensicherheit}\\
	Ein nicht zu vernachlässigender Aspekt ist die Sicherheit der Daten und Unternehmensgeheimnisse. Wenn man die Daten aus der Hand gibt, um aus Source Code wirklich ausführbare Programme zu machen, so gibt man auch sein wertvollstes Gut aus der Hand. Man muss dem Unternehmen, das die Services anbietet schon besonders trauen, beziehungsweise sollte es finanzstark sein, um bei Verstößen gegen Schutzbedürfnisse entsprechende Entschädigungen zu zahlen. Die sicherste Methode ist, den Code nicht aus der Hand zu geben und selbst einen Build Server zu hosten. Wenn das nicht möglich ist, sollte auf einen sehr vertrauenswürdigen Partner geachtet werden.
\end{itemize}
\section{Bekannte Vertreter}
Im Folgenden eine kleine Auswahl an Tools, die jedoch keinen Anspruch auf Vollständigkeit erhebt. Ich habe mich auf bekannte Vertreter beschränkt.
\subsection{Jenkins}
Bei Jenkins handelt es sich um ein OpenSource Produkt. Es wurde 2004 gestartet, war zwischenzeitlich im Besitz von Oracle unter dem Namen Hudson und wurde dann geforkt\footnote{Forken ist eine besondere Art des Abspaltens von Entwicklungszweigen, bei dem ein Projekt in mehreren Folgeprojekten resultiert} unter dem Namen Jenkins, unter dem es heute bekannt ist. 
%Es hat eine sehr breite User Basis und auch sehr viele Unterstützer, die Änderungen beitragen. Auch die Menge an Dokumentation ist aufgrund der breiten Userbasis sehr gut.
\begin{center}
  \begin{tabularx}{\textwidth}{lX}
    \toprule
    Kategorie & Wert \\
    \midrule
    Firma & "`The Jenkins Project"'. Dies ist eine rechtliche Dachorganisation und hält die Marke "`Jenkins"' \\
		\addlinespace
    Kosten & Es ist ein Open Source Projekt. Es fallen keine Kosten für die Software selbst an. Nur die Server auf denen Jenkins betrieben wird müssen bei manchen Angeboten (z.B. Cloud) bezahlt werden. \\
		\addlinespace
		Version & 2.121.1 (LTS) , 2.129 (Weekly) \\
		\addlinespace
		SCM Systeme & GIT, TFVC, PVCS, Mercurial, Subversion,...\\
		\addlinespace
		Erweiterbar & Ja, via Plugin System\\
    \bottomrule
  \end{tabularx}
	\captionof{table}{Jenkins Fakten}
\end{center}
Ein Beispiel für das Userinterface von Jenkins ist in \autoref{fig:Jenkins-sample} zu sehen. 
\begin{figure}[H]
  \centering
  \fbox{\includegraphics[width=\textwidth]{./Images/Jenkins-sample.png}}
  \caption{Jenkins Weboberfläche \cite{Jenkins-Example}}\label{fig:Jenkins-sample}
\end{figure}
\subsection{TeamCity}
TeamCity ist ein kommerzielles Produkt, das vor allem aus der .NET Welt kommt. Es ist in Java geschrieben und wurde bereits 2006 veröffentlicht. Jetbrains ist eine Firma, die viele Entwicklertools anbietet, um die Produktivität zu erhöhen. 
\begin{center}
  \begin{tabularx}{\textwidth}{lX}
    \toprule
    Kategorie & Wert \\
    \midrule
    Firma &  JetBrains s.r.o., registered seat at Na hřebenech II 1718/10, 14700 Prague 4, Czech Republic \\
		\addlinespace
    Kosten & Teamcity ist ein proprietäres Produkt. Es wird pro Build Agent lizenziert. Der Preis liegt bei etwa 300€ pro Build Agent. Open Source Projekte können das Produkt kostenlos nutzen.\\
		\addlinespace
		Version & 2018.1 \\
		\addlinespace
		SCM Systeme & GIT, TFVC, Mercurial, Subversion,...\\
		\addlinespace
		Erweiterbar & Ja, via Plugin System. Jedoch deutlich weniger Plugins vorhanden als bei Jenkins. Man kann auch selbst Plugins entwickeln.\\
    \bottomrule
  \end{tabularx}
	\captionof{table}{TeamCity Fakten \cite{TeamCity-Marketing}}
\end{center}
Ein Beispiel für das Userinterface von TeamCity ist in \autoref{fig:TC-continuous-integration} zu sehen. 

\begin{figure}[H]
  \centering
  \fbox{\includegraphics[width=\textwidth]{./Images/TC-continuous-integration.png}}
  \caption{Teamcity Weboberfläche \cite{TeamCity-Marketing}}\label{fig:TC-continuous-integration}
\end{figure}

\subsection{TFS}
Der Teamfoundation Server ist die kommerzielle CI Lösung von Microsoft. Historisch kommt es aus dem .NET Umfeld, entwickelt sich aber in den letzten Jahren immer weiter in Richtung Open Source und CrossPlatform. So ist zum Beispiel die letzte Version der TFS Build Automatisierung lauffähig auf vielen verschiedenen Platformen wie z.B. Windows, Linux, MacOS.\\
Auch sind große Teile der Build Automatisierung OpenSource und auf GitHub verfügbar.
\begin{center}
  \begin{tabularx}{\textwidth}{lX}
    \toprule
    Kategorie & Wert \\
    \midrule
    Firma &  Microsoft \\
		\addlinespace
    Kosten & TFS ist ein kommerzielles Produkt. Es wird pro Nutzer lizenziert, wobei rudimentäre Aufgaben mit der "`Stakeholder Lizenz"' kostenlos sind. Sobald man auch die Build Funktionalität nutzen will, braucht man eine bezahlte Lizenz.\\
		\addlinespace
		Version & 2018 \\
		\addlinespace
		SCM Systeme & GIT, TFVC. Allgemein sehr limitiert, denkbar ist eine manuelle Anbindung, was aber dann von den Tools nicht besonders gut unterstützt wird.\\
		\addlinespace
		Erweiterbar & Ja, an vielen Stellen. Sowohl auf Server- als auch auf Clientseite erweiterbar.\\
    \bottomrule
  \end{tabularx}
	\captionof{table}{TFS Fakten \cite{TFS-Marketing}}
\end{center}
Ein Beispiel für das Userinterface von TFS ist in \autoref{fig:tfs_release_pipeline} zu sehen. 

\begin{figure}[H]
  \centering
  \fbox{\includegraphics[width=\textwidth]{./Images/tfs_release_pipeline.png}}
  \caption{TFS Weboberfläche \cite{TFS-Marketing}}\label{fig:tfs_release_pipeline}
\end{figure}
\subsection{Travis-CI}
Travis-CI ist ein reiner On-Premise OnlineService. Er erlaubt es automatisch mit jedem GitHib push einen Build zu starten und zu testen.
\begin{center}
  \begin{tabularx}{\textwidth}{lX}
    \toprule
    Kategorie & Wert \\
    \midrule
    Firma &  TRAVIS CI, GMBH \\
		\addlinespace
    Kosten & Travis-CI ist OpenSource. Es ist kostenlos für OpenSource Projekte. Für alle anderen ist es kostenpflichtig.\\
		\addlinespace
		Version & 2.2.1 \\
		\addlinespace
		SCM Systeme & Nur GitHub\\
		\addlinespace
		Erweiterbar & Man kann per yaml den Build Agent so konfigurieren, dass automatisch Compiler usw. installiert werden, aber direkte Erweiterbarkeit gibt es nicht.\\
    \bottomrule
  \end{tabularx}
	\captionof{table}{TFS Fakten \cite{Travis-docu}}
\end{center}
Ein Beispiel für das Userinterface von Travis ist in \autoref{fig:travis} zu sehen. 

\begin{figure}[H]
  \centering
  \fbox{\includegraphics[width=\textwidth]{./Images/travis.png}}
  \caption{Travis Weboberfläche \cite{Travis-build}}\label{fig:travis.png}
\end{figure}
\pagebreak

\chapter{Jenkins}
2004 startete Kohsuke Kawaguchi damit, Jenkins zu implementieren. Damals noch unter dem Namen "`Hudson"'. Eigentlich wollte er nur ein Tool für sich selbst entwickeln, das ihm half, qualitativ hochwertigen Code in das SCM einzufügen. (Foreword of \cite{smart2011jenkins})
\section{Funktionsumfang}
\section{Erweiterungsmöglichkeiten}
In den Gründen wurde AuditTrail erwähnt, hier ein Plugin dazu: \cite{jenkins-audit-trail}

\pagebreak

\chapter{Anwendungsbeispiele}
Für verschiedene Einsatzszenarien sind eventuell auch verschiedene CI Lösungen die richtige Wahl, deshalb möchte ich hier ein paar Szenarien inklusive möglicher Lösungen vorstellen.
\section{Kleines Open Source Projekt mit GitHub als SCM}
Ein kleines Open Source Projekt mit nur wenigen Collaborators\footnote{Dabei handelt es sich um jemanden, der dauerhaft zu einem Projekt beiträgt und Lese- wie auch Schreibrechte auf einem GitHub Repository hat} möchte die Qualität des Codes ständig im Auge behalten. Jedoch kennt sich keiner von ihnen mit einem speziellen CI System aus. Travis-CI bietet für Open Source Projekte eine kostenlose Lösung an, bei der man lediglich eine im YAML Format erstellte Konfigurationsdatei in seinem GitHub Repository erstellen muss und eine Verbindung zu travis-ci.org herstellen.\\
Ein vereinfachter schematischer Ablauf ist in \autoref{fig:Schema-Travis} zu sehen
\begin{figure}[H]
  \centering
  \fbox{\includegraphics[width=\textwidth]{./Images/Schema-Travis.png}}
  \caption{Vereinfachter Travis-CI Prozess}\label{fig:Schema-Travis}
\end{figure}
Der Ablauf hier wurde vereinfacht dargestellt, und ähnelt in weiten Teilen dem generischen Ablauf der in \autoref{fig:Schema-aufbau} dargestellt und anschließend erklärt wurde. Es gibt bei Travis-CI neben dem möglicherweise beschränkten Zugriff auf die Builds selbst und deren Logs auch die Möglichkeit von sogenannten Badges um vom Build Ergebnis Kenntnis zu erlangen. Dies sind kleine Bilder die man einfach über die Readme.MD von GitHub in die eigene Seite einbinden kann und die das Ergebnis des Builds repräsentieren. Auf der rechten Seite in \autoref{fig:OpenConjurerFramework} ist dies zu sehen.
\begin{figure}[H]
  \centering
  \fbox{\includegraphics[width=0.5\textwidth]{./Images/OpenConjurerFramework.png}}
  \caption{GitHub Seite mit Build Badge}\label{fig:OpenConjurerFramework}
\end{figure}
\section{Mittelständisches Unternehmen aus dem Java Umfeld}
Ein mittelständisches Unternehmen möchte eine kosteneffiziente CI-Lösung einführen. Der Grund dafür ist, dass die Qualität und der Zeitraum bis zum Beheben eines kritischen Fehlers in ihrem in Java entwickelten Kernprodukt von den Kunden als mangelhaft empfunden wird. Es handelt sich dabei um ein Produkt, das keinen Regularien unterliegt und deshalb das CI System frei gewählt werden kann. Wichtigste Faktoren sind also die Verlässlichkeit und Stabilität der Lösung sowie ein möglichst geringer Kostenaufwand. Der Code soll in der eigenen Hand bleiben, da man die eigene Geistige Eigentum schützen möchte kommt nur diese Lösung in Frage. Des weiteren müssen etwa 50 bis 100 Entwickler gleichzeitig auf dem System arbeiten können.\\
Man entscheidet sich aufgrund der Rahmenbedingungen für Jenkins als CI Lösung mit mehreren dedizierten Build Agenten. Der Grund ist die große Auswahl an vorhandenen Plugins, die sehr gute Unterstützung des JAVA Entwicklungstoolings sowie das relativ einfache Skalieren. Ein mögliches solches Szenario ist in \autoref{fig:Schema-Jenkins}
\begin{figure}[H]
  \centering
  \fbox{\includegraphics[width=\textwidth]{./Images/Schema-Jenkins.png}}
  \caption{Vereinfachter Jenkins Prozess mit Agents}\label{fig:Schema-Jenkins}
\end{figure}
\section{Großes Unternehmen mit Teams in unterschiedlichen Zeitzonen}
Da ich selbst als Build- \& Configurationmanager bei einem großen Unternehmen arbeite, möchte ich hier dieses System vorstellen.\\
Es handelt sich bei meinem Arbeitgeber um ein Unternehmen, das regulatorischen Rahmenbedingungen unterliegt. Des weiteren gibt es sehr viele (\~ 650) Nutzer des Systems, die sich in unterschiedlichen Zeitzonen befinden. Momentan existieren circa 1400 verschiedene Builds in unserem System. Aufgrund der schieren Menge an Aufgaben gibt es ein dediziertes Configuration Management Team mit 6 Mitarbeitern. Der Technologiestack ist hauptsächlich in der .NET Welt beheimatet und umfasst C\#, C++, C für Desktopapplikationen und Firmware sowie Java und Xamarin für Cloudservices und Apps.\\
Die Entscheidung war vor allem eine, die vom Management getrieben wurde, und es wurde TFS ausgewählt. Der Prozess umfasst mittlerweile neben CI Schritten zusätzlich nachgelagerte Tests, die auf der installierten Applikation ausgeführt werden sowie erste Release Schritte, so dass hier bereits der Übergang zu CD deutlich begonnen hat. Ein sehr reduzierter Überblick ist in \autoref{fig:Schema-Jenkins} zu sehen.
\begin{figure}[H]
  \centering
  \fbox{\includegraphics[width=\textwidth]{./Images/Schema-TFS.png}}
  \caption{Vereinfachter TFS Prozess mit nachgelagerten asynchronen Tests}\label{fig:Schema-TFS}
\end{figure}


%\section{Inhaltliche Hinweise}
%Eine Hausarbeit hält wissenschaftliche Erkenntnisse über einen Gegenstand fest, die man in Auseinandersetzung mit der Fachliteratur gewonnen hat, und teilt diese Erkenntnisse in einer nachvollziehbaren Reihe folge mit. Um der Nachvollziehbarkeit willen sollte die Hausarbeit etwa die Form haben \emph{'Themenbereich – Wissensstand darüber – Feststellung der Problemlage  – Fragestellung –
%Materialauswahl  und  Methodik – Verfolgung  der  Fragestellung  /  Untersuchung  des  Gegenstands – Ergebnis'}. Der Themenbereich  muss  dem  geplanten  Umfang  entsprechend überschaubar sein. (\url{http://www.ats.uni-muenchen.de/studium_lehre/merkblatt-praesentationsformen.pdf})


%\section{Hinweise zur Form}
%\subsection{Formatierung} 

%Diese LaTeX-Vorlage benutzt folgende Formatierung (vgl. \url{http://www.indogermanistik.uni-muenchen.de/downloads/diverses/hausarbeit.pdf}):
%\begin{itemize}
%\setlength{\itemsep}{0pt}
	%\item Schriftart: Linux Libertine O (alternativ: Times New Roman)
	%\item Schriftgröße: 12 pt
	%\item linker und rechter Rand: 3.5 cm, oberer und unterer Rand: 3 cm
	%\item Zeilenabstand: eineinhalbfach (alternativ: 15 pt Zeilenabstand bei 12 pt Schriftgröße)
%\end{itemize}

%\noindent
%Bei Umsetzung der Vorgaben in Word sollten unbedingt Formatvorlagen definiert werden.

%\subsection{Zitierung} 
%\label{sec:cit}

%Die Zitierweise folgt dem Autor-Jahr-System. Der Stellenverweis erfolgt im Text, Fußnoten sollten nur für Erläuterungen und Kommentare verwendet werden. Folgende Hinweise sind der \emph{Language}-Bibliographie-Vorlage von \url{ron.artstein.org} entnommen:\newline

%\noindent
%The \emph{Language} style sheet makes a distinction between two kinds of in-text citations: citing a work and citing an author.
%\begin{itemize}
%\item Citing a work:
%  \begin{itemize}
%    \setlength{\itemsep}{0pt}
%    \setlength{\parsep}{0pt}
%  \item Two authors are joined by an ampersand (\&).
%  \item More than two authors are abbreviated with \emph{et al.}
%  \item No parentheses are placed around the year (though parentheses
%    may contain the whole citation). 
%  \end{itemize}
%\item Citing an author:
%  \begin{itemize}
%    \setlength{\itemsep}{0pt}
%    \setlength{\parsep}{0pt}
%  \item Two authors are joined by \emph{and}.
%  \item More than two authors are abbreviated with \emph{and colleagues}.
%  \item The year is surrounded by parentheses (with page numbers, if
%    present).
%  \end{itemize} 
%\end{itemize}
%To provide for both kinds of citations, \verb+language.bst+ capitalizes on the fact that \verb+natbib+ citation commands come in
%two flavors. In a typical style compatible with \verb+natbib+, ordinary commands such as \verb+\citet+ and \verb+\citep+ produce short
%citations abbreviated with \emph{et al.}, whereas starred commands such as \verb+\citet*+ and \verb+\citep*+ produce a citation with a
%full author list. Since \emph{Language} does not require citations with full authors, the style \verb+language.bst+ repurposes the starred commands to be used for citing the author. The following table shows how the \verb+natbib+ citation commands work with \verb+language.bst+.
%\begin{center}
%  \begin{tabular}{lll}
%    \toprule
%    Command & Two authors & More than two authors \\
%    \midrule
%    \verb+\citet+ & \citet{hale} & \citet{sprouse} \\
%    \verb+\citet*+ & \citet*{hale} & \citet*{sprouse} \\
%    \addlinespace
%    \verb+\citep+ & \citep{hale} & \citep{sprouse} \\
%    \verb+\citep*+ & \citep*{hale} & \citep*{sprouse} \\
%    \addlinespace
%    \verb+\citealt+ & \citealt{hale} & \citealt{sprouse} \\
%    \verb+\citealt*+ & \citealt*{hale} & \citealt*{sprouse} \\
%    \addlinespace
%    \verb+\citealp+ & \citealp{hale} & \citealp{sprouse} \\
%    \verb+\citealp*+ & \citealp*{hale} & \citealp*{sprouse} \\
%    \addlinespace
%    \verb+\citeauthor+ & \citeauthor{hale} & \citeauthor{sprouse} \\
%    \verb+\citeauthor*+ & \citeauthor*{hale} & \citeauthor*{sprouse} \\
%    \verb+\citefullauthor+ & \citefullauthor{hale} & \citefullauthor{sprouse} \\
%    \bottomrule
%  \end{tabular}
%\end{center}
%Authors of \emph{Language} articles would typically use \verb+\citet*+, \verb+\citep+, \verb+\citealt+ and \verb+\citeauthor*+, though they
%could use any of the above commands. There is no command for giving a full list of authors.

%\subsection{Hinweis Literaturverzeichnis}
%Das Literaturverzeichnis dieser Vorlage beinhaltet als Beispielbibliographie die Literaturangaben des \emph{Language} Stylesheets. 


%\subsection{Glossierung}

%Standard für die Erstellung von Interlinearversionen sind die \emph{Leipzig Glossing rules} (\url{https://www.eva.mpg.de/lingua/pdf/Glossing-Rules.pdf}). Die Sprachbeispiele werden durchlaufend nummeriert und sind mit einer Quellenangabe zu versehen, s. Beispiel (\nextx).\footnote{Weitere Optionen zur linguistischen Glossierung mit \emph{ExPex} sind der Dokumentation des Pakets zu entnehmen (\url{http://ftp.uni-erlangen.de/ctan/macros/plain/contrib/expex/expex-doc.pdf}).}

%\ex<ex-lezgian>
%\begingl
%\glpreamble  Lesgisch \citep[207]{haspelmath1993grammar} //
%\gla Gila abur-u-n ferma hamišaluǧ güǧüna amuq’-da-č. //
%\glb now they-{\sc obl}-{\sc gen} farm forever behind stay-{\sc fut}-{\sc neg} //
%\glft 'Now their farm will not stay behind forever.' //
%\endgl
%\xe

%Reffering to example by name: (\getref{ex-lezgian}).

%Sprachbeispiele im Text werden kursiv formatiert, ihre Übersetzung mit einfachen Anführungszeichen, z. B. \emph{ferma} 'farm'.



\pagebreak

\addcontentsline{toc}{chapter}{Literatur}
\pagestyle{fancy}

%%\bibliographystyle{language-dt} %using language.bst
%%\bibliography{Seminararbeit} %bib-filename

\defbibheading{head}{\chapter*{Literatur}}
\printbibheading[heading=head]
\printbibliography[type=book, heading=subbibliography, title={Bücher}]
\printbibliography[type=report, heading=subbibliography, title={Forschungsberichte (Research Papers)}]
\printbibliography[type=inproceedings, heading=subbibliography,  title={Konferenzbeiträge}]
\printbibliography[type=misc, heading=subbibliography,  title={Internet}]

%\nocite{*} %List all bib-entries

\clearpage
\chapter*{Abkürzungsverzeichnis}
\addcontentsline{toc}{chapter}{Abkürzungsverzeichnis}
\begin{multicols}{2}
\begin{acronym}[abr]
\acro{CI}{Continuous Integration}
\acro{CD}{Continuous Delivery}
\acro{SCM}{Source Control Management}
\acro{TFS}{Team Foundation Server}
\end{acronym}
\end{multicols}


%%%Block Masterarbeit:
%\pagebreak
%\subsection*{Erklärung}
%\label{erklaerung}
%\vspace*{0.5cm}
%Hiermit versichere ich, dass ich die vorliegende Hausarbeit selbstständig und \mbox{ohne} fremde Hilfe angefertigt, alle benutzten Quellen und Hilfsmittel angegeben und \mbox{Zitate} als solche kenntlich gemacht habe.\\[0.5cm]
%Ich versichere ferner, dass ich die Arbeit weder für eine Prüfung an einer weiteren Hochschule noch für eine staatliche Prüfung eingereicht habe. \\[1.0cm]
%München, den \today \\[2.0cm]
%\rule{6.0cm}{0.4pt} \\
%%%Ende Block Masterarbeit

%%%Block Masterarbeit:
%\pagebreak
%\subsection*{Lebenslauf}
%\label{lebenslauf}
%\vspace*{0.5cm}

%\begin{table}[h]
%\begin{tabularx}{\textwidth}{ l  X }
%\textbf{Persönliche Daten:} & \\
%  Name: & ...\\
%  Geburtsdatum: & ... \\
%  Geburtsort: & ...\\[1.0cm]
%\textbf{Schulausbildung:} & \\
%  1996 - 2000: & ...\\
%  2000 - 2008: & ...\\[1.0cm]  
%\textbf{Studium:} & \\
%  WS 2008 - 2014: & ...\\[1.0cm]
%\end{tabularx}
%\end{table}

%\noindent
%München, den \today \\[2.0cm]
%\rule{6.0cm}{0.4pt} \\
%%%%Ende Block Masterarbeit

\end{document}
